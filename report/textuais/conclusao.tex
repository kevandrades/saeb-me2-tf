%----------------------------------------------------------------------------------------------------------------
% File : conclusao.tex
%----------------------------------------------------------------------------------------------------------------

% ---
% Conclusão
\chapter{Conclusão}


\begin{itemize}

\item A variável Raça/Cor não exerce influência no tempo de afazeres domésticos.
\item A variável Escolaridade da Mãe exerce influência no tempo de afazeres domésticos.
\item A variável Sexo exerce influência no tempo de afazeres domésticos.
\item A proporção de alunos que gastam tempo com afazeres domésticos, cujo as mães que nunca estudaram se difere com todos os outros níveis escolares que, em grande parte gastam menos de uma hora.
\item O mesmo se observa para as mães com o 5º ano incompleto, no qual o único grau que não se difere é dos alunos que responderam que não sabiam sobre o nível educacional da mãe.
\item Não existe diferença sobre as médias das notas dos estudantes de acordo com o sexo.
\item A média das notas dos estudantes varia de acordo com a Raça/Cor e com o nível de escolaridade da mãe.
\item Há uma clara desigualdade entre as notas por raça quando se compara a branca com as demais, e quase todas as desigualdades derivaram dela.
\item As únicas desigualdades que não foram entre comparações de indivíduos de cor branca com outros foi a comparação entre indivíduos pardos e pretos, e também acomparação entre indivíduos pardos e aqueles que não quiseram declarar.
\item Somente aqueles que declararam não saber a escolaridade da mãe tiveram suas hipóteses de igualdade das médias rejeitadas, indicando diferença entre as notas médias desses estudantes com aqueles cujas mães completaram o EnsinoFundamental I e II, ensino médio e faculdade. A única hipótese de igualdade das notas desses estudantes aceita foi a com estudantes cujas mães nunca haviam estudado.


\end{itemize}
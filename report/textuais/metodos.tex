%----------------------------------------------------------------------------------------------------------------
% File : exemplo.tex
%----------------------------------------------------------------------------------------------------------------

% ---
% Este capítulo, utilizado por diferentes exemplos do abnTeX2, ilustra o uso de
% comandos do abnTeX2 e de LaTeX.

\chapter{Estatísticas}
Este estudo usa Métodos estatísticos para obtenção e comparação de valores,
de modo que se possa criar hipóteses que serão testadas, para que seja possível
afirmar com um grau de certeza o comportamento e uma relação entre as variáveis.

\section{Medidas-Resumo}
As medidas-resumos são a base desse estudo para a compreensão, através de valores,
a identidade dos dados, no qual o foco do estudo é obter utilizando métodos e testes
a ponto de resumí-los e compará-los para identificar supostas hipóteses com relação aos dados.
É possível assemelhar a matemática compreendida na observação em uma visão de possibilidade
válida para o comportamento de tal forma a tentar entender a realidade. Formulas dada por 
{\citeonline[cap.~2, 3 e 11]{morettin2017estatistica}}:
$$x \in X,\forall x \in \{x_1,...,x_n\}$$

\begin{equation}
    Média = \bar{x} = \sum_{i=1}^{n} \frac{x_i}{n}
\end{equation}

\begin{equation}
Mediana = md(X) =\left \{\begin{array}{lll}x_{(\frac{n+1}{2})}, \textbf{se} \; n \; \textbf{ímpar}\\\frac{x_{(\frac{n}{2})} + x_{(\frac{n}{2}+1)}}{2}, \textbf{se} \; n \; \textbf{par}\end{array} \right.
\end{equation}

\begin{equation}
Desvio \; Padrão \; Amostral = S = \sqrt{\frac{\sum_{i=1}^{n} (\frac{x_i}{n} - \bar{x})^2}{n-1}}
\end{equation}


\begin{equation}
    1^o \; Quartil = q_1 = \sum_{i=1}^{k} \frac{n_i}{n} = 25\%
\end{equation}

\begin{equation}
    3^o \; Quartil = q_3 = \sum_{i=1}^{k} \frac{n_i}{n} = 75\%
\end{equation}



\section{Testes de comparação}

Os testes de comparação a seguir, propostos para o estudo, têm como objetivo
avaliar a relação de duas populações através do teste do t-Student (teste T), ou
para mais populações utilizando os teste de Kruskal-Wallis (Teste K) e o
teste de Fisher (ANOVA), no qual se presume a independência entre categorias
propostas para a análise e avalia diferenças substanciais entre as variáveis
relacionadas, com base na amostra do banco de dados dos alunos do
9\textsuperscript{o} ano de 2017.

\subsection{Kruskal-Wallis} 

O teste proposto por \citeonline{kruskal1952use}, utiliza a ideia de ranqueamento dos valores, no qual usa-se para comparar mais de duas populações sem a confirmação dos dados serem ''normais''. Estatística do teste dada por:

$$K = \frac{1}{S^2} \sum^r_{j=1} \frac{R^2_i}{n_i} - \frac{n(n+1)^2}{4}$$

\subsection{Fisher}

O famoso teste da análise de variabilidade (ANOVA) dado por \citeonline{fisher1928general}, aplica para os dados considerados como ''normais'' e com a mesma variância, a comparação das médias das populações sendo como iguais.

\subsubsection{Bartlett}

O teste de \citeonline{bartlett1954note} foi proposto para analisar se as variâncias ($S^2$) das populações são iguais (Homocedasticidade), no qual possibilita a aplicação do teste de Fisher. Estatística dada por:


$$B = \frac{M}{C}$$

onde,

$$M = (n-r)lnS^2_e - \sum_{i=1}^{r} (n_i - 1)lnS_i^2$$  

e

$$C = 1+ \frac{1}{3(r-1)}\left[\sum_{i=1}^{r} \left( \frac{1}{n_i-1}\right) - \left(\frac{1}{n-r} \right) \right]$$




\subsection{t-Student}

O T-student \cite{o1908student} tem como base o teste paramétrico, no qual este estudo usa o métodos de comparação de duas populações cujo as variâncias ($S^2$) são iguais. A estatística de teste é:

$$T = \frac{\bar{X}-\bar{Y}}{S_p\sqrt{1/n_X + 1/n_Y}}$$
onde,

$$S_p^2 = \frac{(n_X-1)S_X^2 + (n_Y-1)S_Y^2}{n_X + n_Y -2}$$





\chapter{Metodologia}

Este estudo parte da hipótese de normalidade das notas, analisada por outros estudos
referentes ao banco de dados do SAEB de 2017 divulgada pelo \citeonline{Saeb2017a)}.
O objetivo desse estudo é relacionar variáveis de uma amostra de 2000 alunos deste
banco de dados e analisar possíveis diferenças substanciais através de testes estatísticos.

Nas relações propostas pelo estudo, foi feita uma junção de três amostras de tamanho 2.000, 
e os valores em branco foram omitidos, gerando uma amostra de 5.271 observações.

O P-Valor é a base desse estudo para a decisão da hipótese, no qual a confirmação da hipótese nula
($H_0$) é avaliada com um grau de significância (ou confiabilidade), considerado na análise.
Se este assumir valores menores que um menos a percentagem de confiança considerada,
há evidência de recusar a hipótese $H_0$ \cite[pag. 364]{morettin2017estatistica}.
A aplicação dos testes de hipóteses para a análise da amostra de 5271 alunos utiliza ferramentas
dos softwares \textsc{R} e \textsc{Python}
com pacotes\footnote{Pacotes externos usados para a manipulação dos dados:\\\textit{R: tidyverse, data.table, reshape2, patchwork, EnvStats, PMCMR, gridExtra; \\Python: pandas}}
para a implementação das análises.


\begin{table}[htb]
\caption{\label{p_valor} Escala de significância de Fisher.}
    \centering
    \begin{tabular}{c|cccccc}
    \toprule
    Evidência & marginal & moderada & substancial & forte & muito forte & fortíssima\\
    \midrule
        P-valor & 0,10 & 0,05 & 0,025 & 0,01 & 0,005 & 0,001\\
    \bottomrule
    \end{tabular}
    \fonte{\citeonline[p. 364]{morettin2017estatistica}.}
\end{table}
